\documentclass{article}
  \usepackage[utf8]{inputenc}
  \usepackage{amsmath}
  
  \title{Calculus & Probability theory}
  \author{brigelpineti }
  \date{April 2018}
  
  \begin{document}
  \maketitle
  \section*{Exercise 6}
  
  
  \textbf{a)} x - $x^3$ = x(1-$x^2$) = x(1+x)(1-x) then we have x $\in$ \{-1, 0, 1\} \\ \\
  \textbf{b)} We have (1-$x^2$) $>$ 0 =$>$ $\vert$ x $\vert$ $<$ 1 meaning that x $\in$ (-1, 1). \\ 
  Thus by resulting on 4 cases: \\
  1) x $<$ -1 =$>$ x $<$ 0, (1-$x^2$) $<$ 0, thus f(x) $>$ 0 \\
  2) x \in (-1, 0) => x < 0, (1-$x^2$) $>$ 0, thus f(x) $<$ 0 \\
  3) x \in (0, 1) => x > 0, (1-$x^2$) $>$ 0, thus f(x) $>$ 0 \\
  4) x $>$ 1 =$>$ x $>$ 0, (1-$x^2$) $<$ 0, thus f(x) $<$ 0
  
  
  \section*{Exercise 7}
  
  3 cases are to distinguished considering the values of \textit{a} and \textit{b} \\
  1) a == b then y = \{a\} \\
  2) a $>$ b then (b-a) $<$ 0 \\
  3) a $<$ b then (b-a) $>$ 0. Keeping in mind that x $\in$ $(0,1)$ $ $ then $ $ y = a + (b-a)x $>$ a and y = a + (b-a)x $<$ a + (b-a) = b.\\ \\
  Therefore, y will take values in (a,b) \\ 
  
  \section*{Exercise 8}
  
  \textbf{a)} f(-x) = 3(-x) - $(-x)^3$ = -3x + $x^3$ and -f(x) = -3x + $x^3$ \\
  Since f(-x) $\neq$ f(x) then the function is \textbf{not even}. \\
  Since f(-x) $\eq$ -f(x) then the function is \textbf{odd}. \\ \\
  \textbf{b)} f(-x) = $\sqrt[3]{(1+x)^2}$ + $\sqrt[3]{(1-x)^2}$ and -f(x) = - $\sqrt[3]{(1-x)^2}$ - $\sqrt[3]{(1+x)^2}$ \\
  Since f(-x) $\eq$ f(x) then the function is \textbf{even}. \\
  Since f(-x) $\neq$ -f(x) then the function is \textbf{not odd}.
  
  \section*{Exercise 9}
  \textbf{a)} To find the domain, the expression inside the square root needs to be greater or equal to zero. Therefore, 7 - $x^2$ $\geq$ 0. In order for the former mentioned condition to be true, -$\sqrt{7}$ $\leq$ x $\leq$ $\sqrt{7}$. \\
  So, \textbf{D(f) = [-$\sqrt{7}$, $\sqrt{7}$]}. \\ \\ 
  Knowing the values of x allowed for the expression inside the square root, it is possible to determine the values y may take =$>$ 0 $\leq$ $\sqrt{7 - x^2}$ $\leq$ $\sqrt{7}$. In f(x) the only difference is right part where \textit{+1} is added leading to 0\textbf{+1} $\leq$ $\sqrt{7 - x^2}$ $\leq$ $\sqrt{7} \textbf{+1}$. In order words, \textbf{R(f) = [1, $\sqrt{7}$ + 1]}. \\ \\
  \textbf{b)} Since in a division operation the denominator needs to be $\neq$ 0 then \\ \textbf{D(f) = R \textbackslash \{0\}}. Additionally, the absolute value will guarantee that f(x) will always evaluate to a positive value without caring about the cases when x is positive or negative. Hence, \textbf{R(f) = (0, +$\infty$)}.
  
  \section*{Exercise 10}
  \textbf{a)} Transform expression to y(cx + d) = ax + b \\
  $>$ ycx + yd = ax + b \{place x on one side to group\} \\ 
  $>$ ycx -ax = b - yd  \{now group by x\} \\
  $>$ x(yc - a) = b - yd \{computer for x\} \\
  $>$ x = $\dfrac{b-yd}{yc-a}$ \\ \\
  Therefore, the inverse function is given by g(x) = $\dfrac{b-xd}{xc-a}$ \\ \\
  \textbf{b)} When d = -a then \textit{g} is equal to \textit{f}. To be noted is that g is defined for all \textit{x} besides x = $\dfrac{a}{c}$. f(x) being equal to $\dfrac{a}{c}$ means that ad = bd = 0, thus by saying that g is defined for the whole range of f.
   
  
  \section*{Exercise 11}
  
  \textbf{a)} For all x $\neq$ 2 : \\ 
  $>$ $\dfrac{x-2}{(x-2)(x+3)}$ \\
  $>$ $\dfrac{x-2}{(x-2)(x+3)}$ \\
  $>$ $\dfrac{1}{x+3}$ \\ \\
  So: 
  $$\lim_{x\to2} \dfrac{x-2}{(x-2)(x+3)} = \lim_{x\to2} \\ \\ \dfrac{1}{x+3} = \dfrac{1}{5}$$
  \textbf{b)} For all x $\neq$ 1 : \\
  $>$ $\dfrac{x^2 -4x + 3}{x^2 + x - 2}$ \\
  $>$ $\dfrac{(x-1)(x-3)}{(x-1)(x+2)}$ \\
  $>$ $\dfrac{x-3}{x+2}$ \\ \\
  So:
  $$\lim_{x\to1} \dfrac{x^2 -4x + 3}{x^2 + x - 2} = \lim_{x\to2} \\ \\ \dfrac{x-3}{x+2} = -\dfrac{2}{3}$$
  
  \end{document}
  