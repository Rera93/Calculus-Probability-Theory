\documentclass{article}
  \usepackage[utf8]{inputenc}
  
  \title{Calculus & Probability theory}
  \author{brigelpineti }
  \date{April 2018}
  
  \begin{document}
  \maketitle
  \section*{Exercise 6}
  \begin{itemize}
  
  \item a) x - $x^3$ = x(1-$x^2$) = x(1+x)(1-x) then we have x \in \{-1, 0, 1\} 
  
  \item b) we have (1-$x^2$) $>$ 0 =$>$ $\vert$ x $\vert$ $<$ 1 meaning that x \in (-1, 1). \\ 
  $Thus by resulting on 4 cases:$ \\
  1) x < -1 => x < 0, (1-$x^2$) $<$ 0, thus f(x) $>$ 0 \\
  2) x \in (-1, 0) => x < 0, (1-$x^2$) $>$ 0, thus f(x) $<$ 0 \\
  3) x \in (0, 1) => x > 0, (1-$x^2$) $>$ 0, thus f(x) $>$ 0 \\
  4) x $>$ 1 =$>$ x $>$ 0, (1-$x^2$) $<$ 0, thus f(x) $<$ 0
  \end{itemize}
  
  \section*{Exercise 7}
  
  3 cases are to distinguished considering the values of \textit{a} and \textit{b} \\
  1) a == b then y = \{a\} \\
  2) a $>$ b then (b-a) $<$ 0 \\
  3) a $<$ b then (b-a) $>$ 0. Keeping in mind that x \in $(0,1)$ $ $ then $ $ y = a + (b-a)x > a and y = a + (b-a)x < a + (b-a) = b.  $ $\\ \\
  $ $ Therefore, y $ $ will $ $ take $ $ values $ $ in $ $ (a,b) \\ 
  
  \section*{Exercise 8}
  
  \textbf{a)} f(-x) = 3(-x) - $(-x)^3$ = -3x + $x^3$ and -f(x) = -3x + $x^3$ \\
  Since f(-x) $\neq$ f(x) then the function is \textbf{not even}. \\
  Since f(-x) $\eq$ -f(x) then the function is \textbf{odd}. \\ \\
  \textbf{b)} f(-x) = $\sqrt[3]{(1+x)^2}$ + $\sqrt[3]{(1-x)^2}$ and -f(x) = - $\sqrt[3]{(1-x)^2}$ - $\sqrt[3]{(1+x)^2}$ \\
  Since f(-x) $\eq$ f(x) then the function is \textbf{even}. \\
  Since f(-x) $\neq$ -f(x) then the function is \textbf{not odd}. 
  
  \end{document}
  